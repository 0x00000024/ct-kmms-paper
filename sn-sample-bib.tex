\begin{thebibliography}{9}

\bibitem{bib1} Y. Chang, J. Bae, N. Kim, JY. Park, SM. Lee and JB. Seo, “Three‐dimensional quadratic modeling and
quantitative evaluation of the diaphragm on a volumetric CT scan in patients with chronic obstructive pulmonary
disease,” Medical Physics, vol. 43, no. 7, pp. 4273-4282, June 2016.

\bibitem{bib2} PA. Hodnett and DP. Naidich, “Fibrosing interstitial lung disease. A practical high-resolution computed
Tomography–based approach to diagnosis and management and a review of the literature,” American journal of respiratory
and critical care medicine, vol. 188, no. 2, pp. 141-149, May 2013.

\bibitem{bib3} Y. Vinogradskiy, R. Castillo, E. Castillo, SL. Tucker, Z. Liao, T. Guerrero and MK. Martel, “Use of
4-dimensional computed tomography-based ventilation imaging to correlate lung dose and function with clinical outcomes,”
International Journal of Radiation Oncology*Biology*Physics, vol. 86, no. 2, pp. 366-371, June 2013.

\bibitem{bib4} SH. Bak, SO. Kwon, SS. Han and WJ. Kim, “Computed tomography-derived area and density of pectoralis
muscle associated disease severity and longitudinal changes in chronic obstructive pulmonary disease: a case control
study,” Respiratory Research, vol. 20, no. 226, pp. 1-12, October 2019.

\bibitem{bib5} K. Tanimura, S. Sato, Y. Fuseya, K. Hasegawa, K. Uemasu, A. Sato, T. Oguma, T. Hirai, M. Mishima and
S. Muro, “Quantitative assessment of erector spinae muscles in patients with chronic obstructive pulmonary disease.
novel chest computed Tomography–derived index for prognosis,” Annals of the American Thoracic Society, vol. 20, no. 226,
pp. 334-341, November 2015.

\end{thebibliography}
